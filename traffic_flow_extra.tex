\documentclass[a4paper,10pt]{article}

\usepackage{amsmath,amssymb,amsfonts,mathtools}
\usepackage{tikz}
\usepackage{hyperref}
\usepackage[top=90pt,bottom=90pt,left=95pt,right=95pt]{geometry}
\usepackage{color}

\title{\bf One-dimensional traffic flow}
\author{Project work in Numerical Analysis 1}

\begin{document}



\section*{6 Numerics and analysis of an upwind discretization$^\ast$}
Consider the new initial condition 
\begin{align}
\label{eq:init_cond}
 u^0(x) = \begin{cases}
           1, &\quad 0 \leq x \leq L/3, \\
           2-(3/L) x, &\quad L/3 \leq x \leq 2L/3,\\
           0, &\quad x \geq 2L/3.
          \end{cases}
\end{align}
together with boundary conditions,
\[
 u_l(t) = +1, \quad u_r(t) = 0, \quad t \geq 0.
\]

Again, use Forward Euler for the time integration, and set the parameters to the values given in Table \ref{params}. Plot both, the solution $u$ and the car density $\rho$ at every time step. Do you observe something strange?

\begin{table}[ht]
\centering
 \begin{tabular}{|c|c|c|c|c|}
  \hline
  $\nu$ & $N$ & $L$ & $t_e$ & $\Delta t$\\
  \hline
  $0.01$ & $100$ & $3.0$ & $5.0$ & $0.003$\\
  \hline
 \end{tabular}
\caption{Parameter configuration.}\label{params}
\end{table}

In this task, we want to avoid the \textit{strangeness} that appears when using your original code. Therefore, note that $u>0$ and replace the central discretization for the first derivative in Burgers' equation by an \textit{upwind scheme}. Use the same parameters as before.
\subsection*{6.1 Analysis of upwind scheme}
Consider the inviscid Burgers' equation, that is $\nu = 0$,
\begin{align}
\label{eq:inviscid}
 \frac{\partial u}{\partial t} + \frac{\partial}{\partial x} \left( \frac{1}{2}u^2 \right) = 0.
\end{align}
The total variation $TV(u)$ is defined as,
\begin{align*}
TV(u^{n}) := \sum_{i} |u_i^{n} - u_{i-1}^{n}|, \quad \text{where } u_i^n = u(x_i,t^n),
\end{align*}
and it is known that if $TV(u^{n+1}) \leq TV(u^{n})$ then a monotonic initial condition \eqref{eq:init_cond} guarantees a monotonic solution $u^n$ and, hence, no \textit{wiggles} can occur. Show that this holds for an upwind discretization of \eqref{eq:inviscid}.

\textit{Hint: Structure your proof in the following way:\\
       \phantom{\indent Hint: }1. Write-down the stencil for $u_i^{n+1}$ using $\mu:=\Delta t/(2 \Delta x)$ and $a^2-b^2 = (a-b)(a+b)$,\\
       \phantom{\indent Hint: }2. Do the same for $u_{i-1}^{n+1}$,\\
       \phantom{\indent Hint: }3. Use that $u_i^n \leq \max_i |u_i^n|$, and $\max_i |u_i^n| \frac{\Delta t}{\Delta x} \leq 1$.}

\subsection*{6.2 Number of cars}
We want to find out after which time all cars left the road? Therefore, integrate the car density $\rho$ with respect to $x$ at all time steps. Use the \textit{Trapezoidal rule} and a threshold for an empty road of \texttt{0.001}.

\end{document}
